%% Greek Abstract
\begin{greek}
\begin{abstract}
Στις μέρες μας παρατηρείται μια αυξανόμενη τάση χρησιμοποίησης παράλληλων
τεχνικών στον προγραμματισμό. Αυτό εξηγείται τόσο από την αύξηση των
επεξεργαστών που περιλαμβάνονται πλέον σε κάθε προσωπικό υπολογιστή, όσο
και από την δημιουργία και ανάπτυξη υπηρεσιών Cloud. Ωστόσο ο παράλληλος
προγραμματισμός εισάγει μια σειρά από σφάλματα τα οποία δεν υπάρχουν στον
σειριακό προγραμματισμό και τα οποία τα συνηθισμένα εργαλεία αποσφαλμάτωσης
δεν μπορούν να αναγνωρίσουν.

Σε αυτή τη διπλωματική μελετάμε τον Concuerror, ένα εργαλείο ελέγχου
προγραμμάτων γραμμένων σε Erlang το οποίο μπορεί να χρησιμοποιηθεί για
αυτόν ακριβώς τον σκοπό. Επίσης εξετάζουμε κατά πόσο ένα τέτοιο εργαλείο
μπορεί να χρησιμοποιηθεί από μεγάλα projects (χιλιάδες γραμμές κώδικα),
τα οποία μπορεί να χρησιμοποιούν αρκετές από τις βιβλιοθήκες του OTP και
να υλοποιούν περίπλοκα πρωτόκολλα επικοινωνίας.

\begin{keywords}
Erlang, concurrency, software testing, model checking, test-driven development
\end{keywords}

\end{abstract}
\end{greek}


%% English Abstract
\begin{abstract}
Concurrent programming has become increasingly popular in the last decade. This
can be explained by the increasing number of multiprocessor personal computers
and the new trend of Cloud computing. Nevertheless, concurrent programming
introduces a number of new errors not seen in linear programming and which
traditional testing tools are largely cannot handle.

In this thesis we study Concuerror, a testing tool for concurrent Erlang
programs, that aims to facilitate the task of detecting and eliminating
concurrency-related errors. We also examine how Concuerror can be used
in practice to test projects with thousands lines of code, which may use
many OTP libraries and implement complex intercommunication protocols.

\begin{keywordseng}
Erlang, concurrency, software testing, model checking, test-driven development
\end{keywordseng}

\end{abstract}


%%Greek Acknowledgements
\begin{greek}
\begin{acknowledgements}
Θα ήθελα να πω ένα μεγάλο ευχαριστώ στον Κωστή Σαγώνα για την διαρκή υποστήριξη
και πολύτιμη καθοδήγηση, η οποία συνέβαλε καθοριστικά στη διαμόρφωση αυτής της
διπλωματικής εργασίας, καθώς και για την εμπιστοσύνη και σεβασμό τον οποίο μου
έδειξε. Επίσης, θα ήθελα να ευχαριστήσω τον Νίκο Παπασπύρου για την πολύ
σημαντική βοήθεια την οποία μου προσέφερε.

Χρωστάω ένα μεγάλο ευχαριστώ στους γονείς μου για την υποστήριξη που μου έχουν
προσφέρει και την εμπιστοσύνη που έδειξαν σε κάθε επιλογή μου.

Τέλος, θέλω να πω ένα μεγάλο ευχαριστώ σε όλους τους φίλους μου που μου
στάθηκαν τα τελευταία χρόνια.
\begin{flushright}Άγγελος Γιάντσιος\end{flushright}

\end{acknowledgements}
\end{greek}
