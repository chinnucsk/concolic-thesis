\chapter{Concolic Testing}\label{ch:concolic_testing}

\section{The Basic Idea}\label{sec:conc_basic_idea}

The word "concolic" is a portmanteau of the words concrete and symbolic and it
depicts the hybrid nature of the concolic software testing technique. Concolic testing
interleaves concrete execution with symbolic execution in order to achieve high
path coverage. The basic idea is to use symbolic execution to generate inputs
that will hopefully steer the program to alternate execution paths whilst
concrete execution is used to guide the symbolic execution along a concrete
path.

During the execution, we keep a concrete and a symbolic state. The concrete
state maps the variables to their concrete values whereas the symbolic state
maps the variables that have non-concrete values to their symbolic
representation. In order to start the process we need a seed concrete input,
that is either provided by the user or is randomly generated. We then execute
the program with this input and gather symbolic constraints at all the
conditional statements along the execution. Each constraint is essentially a
logical formula which is expressed as a function of the input symbolic
variables and denotes the boolean value of the respective conditional statement
during the execution. The conjuction of all the constraints is the path
predicate. The next step is to feed the path predicate to a constraint solver
in order to infer new variants of the input that will likely guide the program
to a different feasible execution path. We, then, use the generated input as
our new seed. This process is repeated systematically until every feasible
execution path is explored.

In the concurrent version, a new path may be explored by producing a different 
interleaving of the program threads instead of generating a different input. 
This instrumentation records and permutes the various events that lead to 
data and lock races.

\subsection{A Simple Example}\label{subsec:conc_simple_example}

Consider the following simple Erlang function (Listing~\ref{lst:concolic_example}) 
that has an erroneous branch:

\includecode[erlang]{lst:concolic_example}{Simple Erlang function}{%
  listings/concolic_example.erl
}

Any black-box testing technique would most likely require a lot of time, if ever, 
to locate and reproduce the failure. 

Let's see how concolic testing will help locate promptly the error. As we said, 
we need a seed input to initialize our search. We arbitrary choose the input 
\texttt{X = 0} and \texttt{Y = 0} and execute the code with these concrete 
values. In line 2, we get \texttt{Z = 2 * Y = 0}. In line 3, the condition 
evaluates to \texttt{false} as \texttt{X =/= 100000} and 
\texttt{erlang:andalso/2} is short-circuited. Therefore, we follow the 
"false" branch and the function returns \texttt{ok} and ends normally. At the 
same time we execute the function symbolically following the same path as the 
concrete execution. This yields the path constraint \texttt{(X =/= 100000)}.

In order to force the program to follow a different a execution path in the next 
iteration, we negate the only predicate in the patch constraint. This yields 
the binding \texttt{X = 100000}, so we execute the function again with the 
concrete values \texttt{X = 100000} and \texttt{Y = 0}. In line 2, we still get 
\texttt{Z = 0}. In line 3, the expression \texttt{X =:= 100000} evaluates 
to \texttt{true} but the expression \texttt{X < Z} evaluates to \texttt{false}, 
thus making the whole condition evaluate to \texttt{false} again. The function 
again returns \texttt{ok} and ends normally. After the symbolic execution we 
get the path constraint \texttt{(X =:= 100000) $\land$ (X >= 2 * Y)}.

This time we invoke the solver to negate the second predicate while keeping 
the first predicate true. The solution may be possibly be \texttt{X = 100000} 
and \texttt{Y = 50001}. By running the function with this concrete input, we 
manage to reach the branch that raises the exception.

As we see, this approach manages to locate the error in only three iterations.

\subsection{Strengths and Limitations}\label{subsec:concolic_strength_limitation}

\subsection{Challenges in Erlang}\label{subsec:concolic_chalenges_in_erlang}

\subsection{Z3 SMT Solver}\label{subsec:z3_smt_solver}

The theorem prover we chose to use is Z3 from Microsoft Research. It is an 
efficient SMT solver that is targeted at solving problems in software analysis 
and software verification. It supports the SMT-LIBv2 standard and it provides 
APIs for C/C++, .NET, OCaml and Python. We used the Python API, namely Z3Py.

We should note that we used it under the Microsoft Research License Agreement for 
Non-Commerical use.


